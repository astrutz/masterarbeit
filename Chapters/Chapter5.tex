%!TEX root = ../main.tex
% Chapter 5

\chapter{Definition von Nutzeranforderungen}
\label{Definition_von_Nutzeranforderungen}

Nachfolgend werden die in Kapitel \ref{Operative_Ziele} definierten operativen Ziele in Nutzeranforderungen überführt. Darüber hinaus werden aus den Interessen, Ansprüchen, Anteilen und Anrechten der Stakeholder aus Kapitel \ref{Stakeholderanalyse} weitere Anforderungen definiert. Die Anforderungen werden im User Story Format von \cite{Cohn2004} formuliert und beinhalten eine Beschreibung aus Nutzersicht, eine Erklärung des Zwecks, sowie eine darauf folgende Liste an Akzeptanzkriterien. Darüber hinaus wird die Priorität mit 1 (niedrig) bis 3 (hoch) angegeben. Die Anforderungen dienen als Basis für die technische und inhaltliche Konzeption, sowie die spätere Implementation des Systems.


%----------------------------------------------------------------------------------------

\section{Anforderungen von Absendern}
\label{Anforderungen_von_Absendern}

\textbf{Einsicht des Nachrichtenaufkommens beim Empfänger} \\
Priorität: 1 \\
Als Absender möchte ich die Anzahl der zu bearbeitenden E-Mails beim Empfänger einsehen können, um einzuschätzen wie lange ich auf eine potenzielle Antwort warten muss.
\begin{itemize}
    \item Die Anzahl der offenen Nachrichten ist dem Absender bekannt und kann eingesehen werden.
    \item Persönliche Daten und Informationen werden dem Absender nicht übermittelt.
    \item Die voraussichtliche Antwortzeit wird, abhängig von der Bearbeitungsleistung (s.u.) des Empfängers, angezeigt.
\end{itemize}

\noindent
\\ \textbf{Priorisierbarkeit von E-Mails mit Gegenwert} \\
Priorität: 1 \\
Als Absender möchte ich eine E-Mail mit einem endlichen Gegenwert versehen können, um eine schnellere Bearbeitung gewährleistet zu bekommen.
\begin{itemize}
    \item E-Mails können optional mit einem Gegenwert versehen werden, der für Absender und Empfänger einsehbar ist.
    \item Dem Gegenwert wird eine Bearbeitungszeit zugeordnet, die von Absender und Empfänger einsehbar ist.
    \item Die Bearbeitungszeit ist verbindlich und muss berücksichtigt werden.
\end{itemize}

\noindent 
\textbf{Technologisch unabhängige Nutzung} \\
Priorität: 2 \\
Als Absender möchte ich meinen bevorzugten E-Mail Client verwenden können, um technologisch flexibel zu sein.
\begin{itemize}
    \item Die Einsicht in das Nachrichtenaufkommen und die Priorisierung von E-Mails sind clientunabhängig und können von jeglichem Endgerät aus genutzt werden.
    \item Die Priorisierung wird unmittelbar im E-Mail Standard umgesetzt.
    \item Die zusätzlich nötige Implementation kann von jedem Client ohne weitere Schritte eingebunden werden.
\end{itemize}

\noindent 
\\ \textbf{Sicherheit der E-Mails und Zahlungen} \\
Priorität: 2 \\
Als Absender möchte ich meine E-Mails, sowie Zahlungsinformationen sicher übertragen, um Missbrauch zu verhindern.
\begin{itemize}
    \item Der Inhalt von E-Mails, sowie Informationen über das Nachrichtenaufkommen werden verschlüsselt übertragen und sind von außen nicht einsehbar.
    \item Zahlungsvorgänge und -informationen werden verschlüsselt übertragen und sind von außen nicht einsehbar.
\end{itemize}

\noindent 
\\ \textbf{Keine Verlangsamung des E-Mail Prozesses} \\
Priorität: 3 \\
Als Absender möchte ich meine aktuelle Geschwindigkeit beim E-Mail Versand beibehalten, um meine Produktivität nicht einzuschränken.
\begin{itemize}
    \item Die Einsicht in das Nachrichtenaufkommen verlangsamt den Prozess des E-Mail Versands nicht signifikant.
    \item Die Priorisierung von E-Mails verlangsamt den Prozess des E-Mail Versands nicht signifikant.
\end{itemize}

%----------------------------------------------------------------------------------------

\section{Anforderungen von Empfängern}
\label{Anforderungen_von_Empfaengern}


\textbf{Automatische Triage von E-Mails} \\
Priorität: 1 \\
Als Empfänger möchte ich priorisierte E-Mails angezeigt und sortiert erhalten, um meine persönliche Triage zu verkürzen.
\begin{itemize}
    \item Priorisierte E-Mails werden nach ihrem Gegenwert sortiert angezeigt.
    \item Priorisierten E-Mails wird ein Fälligkeitsdatum, abhängig von der eigenen Bearbeitungsleistung (s.u.), hinzugefügt, welches verbindlich ist.
    \item E-Mails ohne Gegenwert werden nicht sortiert und erhalten kein Fälligkeitsdatum.
\end{itemize}

\noindent
\\ \textbf{Definition der Bearbeitungsleistung} \\
Priorität: 2 \\
Als Empfänger möchte ich meine Bearbeitungsleistung hinterlegen können, um die Triage transparenter zu gestalten und Absender zu informieren.
\begin{itemize}
    \item Es kann eine Bearbeitungsleistung definiert werden, die die Anzahl der vom Empfänger zu bearbeitenden E-Mails pro Tag beschreibt.
    \item Die Bearbeitungsleistung ist für den Empfänger verbindlich und muss eingehalten werden.
    \item Eine Abweichung von der Bearbeitungsleistung und die damit entstehende Verzögerung wird dem Absender automatisch mitgeteilt.
\end{itemize}

\noindent
\\ \textbf{Integrität der E-Mails und Zahlungen} \\
Priorität: 2 \\
Als Empfänger möchte ich bei priorisierten E-Mails sichergehen können, dass der  Gegenwert auch tatsächlich hinterlegt wurde.
\begin{itemize}
    \item Der Gegenwert einer E-Mail kann nicht anhand ihres Headers oder Bodys von außen ausgelesen werden.
    \item Es ist nicht möglich Zahlungen zu fälschen oder den Gegenwert einer E-Mail zu verändern.
    \item Zahlungen sind an einer nicht kompromittierbaren Stelle gespeichert und werden beim Empfang von E-Mails abgeglichen.
\end{itemize}

\noindent
\\ \textbf{Abweichungen der automatischen Triage} \\
Priorität: 3 \\
Als Empfänger möchte ich E-Mails nachträglich sortieren und priorisieren können, um für mich relevante Nachrichten, die keinen Gegenwert erhalten haben, schneller bearbeiten zu können.
\begin{itemize}
    \item E-Mails können nach der Triage höher priorisiert werden, als sie es eigentlich sind.
    \item Es können Regeln definiert werden (z.B. anhand bestimmter Absender oder gewisser Begriffe im Betreff), nach welchen E-Mails nach der eigentlichen Triage höher priorisiert werden.
    \item Von der Triage abweichende E-Mails erhalten kein Fälligkeitsdatum.
    \item Von der Triage abweichende E-Mails, die bearbeitet werden, werden nicht zur täglichen Bearbeitungsleistung hinzu gezählt.
\end{itemize}

\newpage
\section{Übergreifende Nutzeranforderungen}

\textbf{Erreichbarkeit} \\
Priorität: 1 \\
Als Nutzer möchte ich die Priorisierung und Einsicht jederzeit nutzen können, um sie verlässlich in meinen E-Mail Prozess zu integrieren.
\begin{itemize}
    \item Die Priorisierung und Einsicht sind jederzeit verfügbar.
    \item Bei Nichterreichbarkeit sind Alternativen formuliert, die zumindest einen Teil der o.g. Anforderungen erfüllen.
\end{itemize}

\noindent
\\ \textbf{Koexistenz zu bisherigen Kanälen} \\
Priorität: 2 \\
Als Nutzer möchte ich meine bisherigen Kommunikationskanäle weiter nutzen können, um unabhängig vom System zu bleiben.
\begin{itemize}
    \item Kommunikationskanäle außerhalb der E-Mail bleiben unverändert.
    \item E-Mails können weiter im herkömmlichen Sinne verwendet werden, die Priorisierung ist optional.
\end{itemize}

\noindent
\\ \textbf{Intuitive Nutzung} \\
Priorität: 2 \\
Als Nutzer möchte eine der bisherigen E-Mail ähnlichen Oberfläche erhalten, um einen möglichst einfachen Übergang zu haben.
\begin{itemize}
    \item Die Benutzeroberfläche ähnelt bisherigen E-Mail Clients.
    \item Die Benutzeroberfläche ist einfach verständlich und bietet einen schnellen Umstieg.
    \item Es ist möglich seinen E-Mail Prozess ohne einen großen Aufwand umzustellen.
\end{itemize}