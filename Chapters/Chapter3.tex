%!TEX root = ../main.tex
% Chapter 3

\chapter{Formulierung der Zielhierarchie}
\label{Formulierung_der_Zielhierarchie}

Zur Lösung des Problemfeldes, der Beantwortung der Forschungsfrage, sowie Berücksichtigung der Erkenntnisse der Stakeholderanalyse wird eine Zielhierarchie bestimmt, die den finalen Zustand des Prozesses \textit{E-Mail Versand} abbildet. Diese Hierarchie gliedert sich in strategische, taktische und operative Ziele, welche aufeinander aufbauen und sich gegenseitig referenzieren. Die operativen Ziele bilden zusammen mit der Stakeholderanalyse die Grundlage für die Definition von Nutzeranforderungen in Kapitel \ref{Definition_von_Nutzeranforderungen}. Die Priorität der einzelnen Ziele wird durch die Stichwörter \textit{muss} (hohe Priorität), \textit{soll} (mittlere Priorität) und \textit{kann} (niedrige Priorität) festgelegt.

%----------------------------------------------------------------------------------------

\section{Strategische Ziele}
\begin{enumerate}[label=(\alph*)]
    \item \textbf{Vereinfachung der Triage}\\
        Der Prozess der Triage muss für den Empfänger vereinfacht werden, um den E-Mail Overload zu verringern und einhergehende Folgen wie einen erhöhten Stress zu minimieren.
        
    \item \textbf{Einfluss durch Absender}\\
        Der Einfluss, den ein Absender auf die Beantwortung seiner E-Mail nehmen kann, muss eindeutig definiert und verbindlich sein.
\end{enumerate}

%----------------------------------------------------------------------------------------

\section{Taktische Ziele}
\begin{enumerate}[label=(\alph*)]
\setcounter{enumi}{2}
    \item \textbf{Erreichbarkeit}\\
    \textit{Referenziert Ziel a} \\
        Die bisherige Erreichbarkeit von Empfängern muss gewährleistet bleiben.
        
    \item \textbf{Triage ohne Interaktion des Empfängers}\\
    \textit{Referenziert Ziel a} \\
        Die Triage von E-Mails muss mit einem automatisierten Verfahren umgesetzt werden, um die Aufgabenlast des Empfängers zu verringern.
        
    \item \textbf{Transparente Warteschlange}\\
    \textit{Referenziert Ziel b} \\
        Die Anzahl der unbearbeiteten E-Mails, sowie ihre Priorität sollen dem Absender transparent dargestellt werden, um eine Aussage über die Priorität der eigenen E-Mail treffen zu können.
    
    \newpage
    
    \item \textbf{Einfluss mit Gegenwert}\\
    \textit{Referenziert Ziel b} \\
        Der Einfluss der Absenders muss stets mit einem Gegenwert behaftet sein, um eine nachweisbare Relevanz zu zeigen. Ohne einen Gegenwert wäre der Einfluss mit dem E-Mail Header-Feld \textit{Priorität} vergleichbar und somit nicht verbindlich.
\end{enumerate}

%----------------------------------------------------------------------------------------

\section{Operative Ziele}
\label{Operative_Ziele}
\begin{enumerate}[label=(\alph*)]
\setcounter{enumi}{6}
    \item \textbf{Produktivität von Absender und Empfänger}\\
    \textit{Referenziert Ziel c} \\
        Der bisherige Grad der Produktivität soll sich weder beim Empfänger noch beim Absender verringern, um einen zeitlichen Vorteil im Vergleich zur manuellen Triage zu schaffen. 
        
    \item \textbf{Weitere Kommunikationswege}\\
    \textit{Referenziert Ziel c} \\
        Kommunikationswege außerhalb der E-Mail sollen unberührt bleiben und können im selben Maß genutzt werden. 
        
    \item \textbf{E-Mail Standard}\\
    \textit{Referenziert Ziel c} \\
        E-Mails müssen als Medium weiterhin nutzbar bleiben. Da Absender kein Anrecht, sondern lediglich ein Interesse an einer zeitnahen Beantwortung von herkömmlichen E-Mails haben, ist die Entwertung der klassischen E-Mail vertretbar. Siehe dazu auch Kapitel \ref{Stakeholderanalyse}.
    
    \item \textbf{Priorisierbare E-Mail}\\
    \textit{Referenziert Ziel d} \\
        E-Mails, die in einer automatischen Triage verarbeitet werden, müssen vom Absender priorisierbar sein. Diese Priorität soll in verschiedenen Stufen verfügbar sein, um die Genauigkeit der Triage sicherzustellen.
         
    \item \textbf{Priorisierte Darstellung}\\
    \textit{Referenziert Ziel d} \\
        E-Mails müssen dem Empfänger nach Priorität aufgelistet dargestellt und sortiert werden, um die vom Absender vorgesehene Bearbeitung sicherzustellen. 
        
    \item \textbf{Abweichungen der Priorität}\\
    \textit{Referenziert Ziel e} \\
        Empfänger sollen gewisse Abweichungen von der automatischen Triage vornehmen können, um E-Mails zu berücksichtigen, die beispielsweise für sie aber nicht für den Absender eine hohe Relevanz aufweisen.
        
    \item \textbf{Einsehen des E-Mail Aufkommens}\\
    \textit{Referenziert Ziel e} \\
        Empfänger müssen das generelle, sowie das akute Aufkommen von E-Mails eines Empfängers einsehen können, um einzuschätzen welcher Gegenwert angebracht ist. Dabei müssen personenbezogene Daten, sowie der Inhalt der Nachrichten entfernt oder anonymisiert werden. 
    
    \newpage
        
    \item \textbf{Voraussichtliche Antwortzeit}\\
    \textit{Referenziert Ziel e} \\
        Absender können die voraussichtliche Antwortzeit einsehen, sodass sie diese nicht auf anderen Kommunikationskanälen erfragen müssen und ihrerseits entsprechende Aussagen gegenüber Vorgesetzten o.ä. treffen können. 
        
    \item \textbf{Endlicher Gegenwert}\\
    \textit{Referenziert Ziel f} \\
        Absender müssen priorisierte E-Mails mit einem endlichen Gegenwert, wie beispielsweise Geld oder Token, gewichten können, um eine verbindliche Relevanz zu verdeutlichen.
        
    \item \textbf{Verfügbarkeit des Gegenwerts}\\
    \textit{Referenziert Ziel f} \\
        Der Gegenwert muss jedem Absender zu Verfügung stehen können, sodass keine Nutzer benachteiligt werden.
\end{enumerate}