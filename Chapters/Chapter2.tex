%!TEX root = ../main.tex
% Chapter 2

\chapter{Stakeholderanalyse}
\label{Stakeholderanalyse}

In diesem Kapitel werden die Stakeholder am E-Mail Verkehr in Unternehmen aufgelistet. Zusätzlich wird ihre Beziehung zum Prozess mit dem entsprechenden Objektbereich abgebildet. Die Beziehung richtet sich nach ISO 15288, nach der ein Stakeholder \emph{\glqq ein Anrecht, einen Anteil, einen Anspruch oder ein Interesse auf ein bzw. an einem System oder an dessen Merkmalen hat\grqq} \citep{ISO2015}. Die Priorität auf den Prozess wird mit den Werten 1 bis 3 dargestellt, wobei 1 ''hoch'' und 3 ''niedrig'' entspricht.

Die Stakeholderanalyse ist in Tabelle \ref{tab:stakeholderanalyse} visualisiert. Als wichtigste Stakeholder werden Absender und Empfänger definiert. Hierbei handelt es sich nicht um feste Personen, sondern um Rollen, welche je nach Kontext wechseln können. So kann ein Empfänger zum Absender innerhalb der gleichen E-Mail werden, wenn er sie z.B. weiterleitet.

Der Absender hat ein Anrecht auf einen unbegrenzten Versand von E-Mails. Das bedeutet, dass er durch keine technischen oder organisatorischen Richtlinien beschränkt wird. Darüber hinaus besitzt er ein Interesse daran, dass seine E-Mail, falls nötig, beantwortet werden. Da es im Ermessen des Empfängers liegt, ob eine Antwort versendet wird, kann nicht von einem Anrecht gesprochen werden. Selbiges gilt für eine Antwortzeit. So kann ein Absender zwar eine Antwortzeit definieren und erwarten, sie letztendlich jedoch nicht durchsetzen.

Um die E-Mail als zuverlässiges Kommunikationsmittel nutzen zu können, hat der Empfänger ein Anrecht auf Erreichbarkeit. Außerdem benötigt er freie Zeiteinteilung bei der Triage von E-Mails. Er ist an E-Mails interessiert, die strukturiert sind und eine einfach erkennbare Priorität ausweisen. Da diese Faktoren abhängig vom Absender sind, kann kein Anspruch oder Anrecht geltend gemacht werden. Zusätzlich hat der Empfänger ein Interesse an einer Benachrichtigung von neuen E-Mails.

Weiterhin ist das Management, bzw. Vorgesetzte zu benennen, die den Anspruch haben, dass Mitarbeiter trotz der Bearbeitung von E-Mails weiterhin produktiv sind. Trotz dessen bleibt auch der Anspruch bestehen, dass Mitarbeiter über E-Mail für Fragen erreichbar sind. Diese Ansprüche können sie, falls gewünscht, durch Priorisierung von Aufgaben oder Eskalationsstufen wie Anweisungen, Forderungen und Mahnungen durchsetzen.

Weitere indirekte Stakeholder beinhalten Systemadministratoren und \acrfull{isp}, die einen Anteil an der Kommunikation per E-Mail haben, indem sie die Infrastruktur bereitstellen und Schutzmaßnahmen vor Spam und Angriffen ergreifen. Ferner haben Gewerkschaften ein Interesse daran, dass der E-Mail Overload Angestellte nicht belastet und sie sich nur während ihrer bezahlten Arbeitszeit mit geschäftlichen E-Mails auseinandersetzen.   
