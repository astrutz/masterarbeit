%!TEX root = ../main.tex
% Chapter 13

\chapter{Fazit}
\label{Schluss}
Dieses Kapitel schließt die Arbeit ab, indem die durchgeführten Schritte und die wichtigsten Erkenntnisse zusammengefasst werden. Anschließend werden die Ergebnisse kritisch bewertet, gefolgt von der darauf aufbauenden Beantwortung der Forschungsfrage. Abgerundet wird das Kapitel durch einen kurzen Ausblick, der Erweiterungen von Pay2Mail und mögliche Entwicklungen im Themenfeld \textit{E-Mail Overload} thematisiert.

\section{Zusammenfassung}
\label{Zusammenfassung}
Zu Beginn wurde das Thema \textit{E-Mail} im Allgemeinen eingeleitet mit dem Verweis, dass die Anzahl der versendeten E-Mails stetig steigt. Das generell hohe Aufkommen von E-Mails führt dazu, dass Nutzer eine mehrschrittige Triage durchführen müssen, um erhaltene Nachrichten zu priorisieren. Eine Triage ist zeitaufwändig und kann fehlerhaft durchgeführt werden. Das führt dazu, dass dringende E-Mails nicht rechtzeitig beantwortet werden, die Produktivität von Mitarbeitern, die mit E-Mails arbeiten, leidet und Nutzer eine Überforderung bei der Bearbeitung von E-Mails, dem E-Mail Overload, wahrnehmen. Eine Verbesserung der Triage mit Einfluss des Absenders wird als Forschungsfrage formuliert. In der folgenden Stakeholderanalyse wurden Absender und Empfänger als wichtigste Stakeholder definiert und werden somit im Weiteren behandelt. Daher beziehen sich die formulierten strategischen Ziele \textit{Vereinfachung der Triage} auf den Empfänger und \textit{Einfluss durch den Absender} auf den Absender. Nach der Definition einer Zielhierarchie wurden die inhaltlichen Grundlagen, die für die Arbeit relevant sind, behandelt. In erster Linie wurde die Technologie E-Mail mit Bezugnahme auf die verschiedenen RFCs tiefer gehend erklärt. Außerdem wurde das Konzept von Pay to Win inklusive der verschiedenen Zahlungsarten und Auswirkungen erläutert. Aus den operativen Zielen der Zielhierarchie wurden Nutzeranforderungen im User Story Format definiert, die die Erfüllung der Ziele abdecken.

In der Recherche wurden Produkte und Forschungsansätze diskutiert, die zumindest einen Teilbereich des E-Mail Overload abdecken. Im Fokus stand dabei \textit{Payment-Based Email} von \cite{Turner2003}. Dieses System versucht Spam-Nachrichten zu minimieren, indem SMTP um einen \texttt{PAYMENT}-Befehl erweitert wird, der die Echtheit einer E-Mail verifizieren soll. Auch wenn dieses Konzept nicht implementiert wurde, bot es die Grundlage für die zahlungsorientierte Priorisierung von E-Mails. Statt der Erweiterung des SMTP wurde ein neuer E-Mail Header namens \texttt{X-Pay2Mail-Priority} definiert, welcher Zahlungsinformationen enthält. In der inhaltlichen Konzeption wurden mögliche Formen von Zahlungen und Gegenwerten gesammelt. Außerdem wurden die Navigation durch das System, sowie die Inhalte modelliert. Die Inhalte wurden in der technischen Konzeption zu einem Datenmodell und einem Komponentenmodell erweitert. Außerdem wurde Ruby on Rails als Fullstack Framework zur Entwicklung des Systems ausgewählt.

Vor der Entwicklung des eigentlichen Systems wurde ein Prototyp implementiert, der die Priorisierung einer E-Mail mit einer Zahlung simuliert. Dieser Prototyp wurde 39 Personen vorgestellt, die sich mit 61,5\% für eine Zahlung mit Echtgeld, mit 23,1\% für ein Tokensystem und mit 15,4\% gegen eine Zahlung im Allgemeinen entschieden. Es wurde ein Zusammenhang zwischen dem Jahresbruttoeinkommen und der Bereitschaft für E-Mails mit Echtgeld zu zahlen hergestellt. Aufgrund der absoluten Mehrheit wurde sich für Echtgeldzahlungen entschieden. Die Entwicklung des Systems wurde in die drei Arbeitspakete \textit{Einsicht in das E-Mail Aufkommen}, \textit{Priorisierung von E-Mails} und \textit{Automatische Triage} unterteilt, die nacheinander bearbeitet wurden. Ergebnis ist eine produktive Webanwendung unter \url{https://pay2mail.org}. In der abschließenden Nutzerevaluation wurde das System je zwei Absendern und Empfängern vorgestellt. Sie gaben an, dass das System eine Unterstützung in der Bearbeitung von E-Mails ist und dem Problem des E-Mail Overload entgegenwirken, es aber nicht selbstständig lösen kann.

\section{Reflexion und Bewertung}
\label{Reflektion_und_Bewertung}
Die in dieser Arbeit und in Kapitel \ref{Vorgehensweise} angestrebte Vorgehensweise ist generell als positiv zu bewerten. Die Herleitung von Zielen und Nutzeranforderungen aus dem relevanten Problemfeld heraus impliziert eine entsprechende Gewichtung des Themas. Durch die Beschränkung auf E-Mail Overload und absendergesteuerte Priorisierung war gewährleistet, dass das Handlungsfeld nicht zu groß wurde und eine gezielte Lösung konzeptioniert werden konnte. Der Rechercheprozess mit späterer Fokussierung auf \textit{Payment-Based Email} unterstützte maßgeblich in der Lösungsfindung. Auch wenn es sich nicht um eine konkrete Weiterentwicklung handelte, bot das Werk die konzeptionelle und technische Basis. Durch die Befragung des Zahlungssystems wurde sichergestellt, dass spätere Nutzer an dem System interessiert sind, da es ihre favorisierte Form des Gegenwertes abdeckt. Allerdings ist zu beachten, dass die Implementation des Prototyps und die Befragung einen Einfluss auf den Umfang des finalen System haben. Wären diese Schritte ausgelassen worden, hätte das System eine Anbindung an reale Zahlungsdienstleister erhalten können. Zu kritisieren ist auch die einmalige Nutzerevaluation, die zwar wichtiges Feedback der Nutzer sammelte, aber keine Möglichkeit gab es im Entwicklungsprozess umzusetzen. Mit der Nutzung des menschzentrierten Gestaltungsprozesses nach ISO 9241-210 oder eines ähnlichen Ansatzes hätte das Feedback der Nutzer, auf Kosten der Entwicklungszeit, besser verwertet werden können.

Auch das entwickelte System \textit{Pay2Mail} kann grundlegend als positiv bewertet werden. Es besitzt einen gewissen Reifegrad und konnte sich sowohl in der Befragung zum Zahlungssystem, als auch in der Evaluation als nutzbar und zum größten Teil intuitiv darstellen. Die Priorisierung von E-Mails durch Zahlungen und das darunter liegende Bietersystem wurde von den Nutzer akzeptiert und verstanden. Auch die Triage wurde von den Empfängern genutzt und individuell erweitert. Am kritischsten ist der soziale Aspekt des Systems zu bewerten. Während der finanzielle Gegenwert einer E-Mail ihre Relevanz definiert, trifft dies nicht auf Empfänger zu. Es lässt sich feststellen, dass reiche Absender nicht zwingend relevantere E-Mails als arme Absender versenden. Diese soziale Ungerechtigkeit führt auch dazu, dass reiche Absender die Priorisierung missbrauchen können und eine schnelle Bearbeitung als Luxus erwerben können, auch wenn eine E-Mail keiner schnellen Bearbeitung bedarf. Darunter leiden ärmere Absender, die mit großen Zahlungen nicht mithalten können und somit ihre E-Mails nicht priorisieren können. Um dieses Problem zu lösen, können verschiedene Ansätze gewählt werden. In Kapitel \ref{Ergebnisse der Evaluation} wird vorgeschlagen, dass Empfänger selbst Absender priorisieren, die sich den Gegenwert nicht leisten können. Das setzt allerdings voraus, dass der Empfänger sich über die finanzielle Situation seiner Absender im Klaren ist. Alternativ könnten finanziell schwächere Absender eine Anzahl an Token beantragen, die empfängerübergreifend gelten. Der Wert der Token könnte sich dabei an den Durchschnitten der gesetzten Gegenwerte eines Empfängers richten. Diese Token stehen in keinem Widerspruch zu den Ergebnissen der Befragung, da Token nur an Absender ausgeteilt werden würden, die jene benötigen. Wenn die Token gesetzt werden, können sie direkt in Echtgeld umgerechnet werden, sodass andere Absender sie nicht bemerken. Ein solches bedarfsorientiertes Tokensystem deckt sich mit den Befragungsergebnissen (siehe Abbildung \ref{fig:Auswahl_des_Zahlungssystems_der_Befragten_nach_Jahresbruttoeinkommen}) und kann das Risiko der sozialen Ungerechtigkeit von Pay2Mail eindämmen.

\section{Beantwortung der Forschungsfrage}
\label{Beantwortung_der_Forschungsfrage}
Die in Kapitel \ref{Forschungsfrage} gestellte Forschungsfrage lautet:

\begin{quotation}
	\noindent Wie kann die E-Mail Triage so verändert werden, dass sie den Aufwand des Empfängers verringert und gleichzeitig einen Einfluss des Absenders ermöglicht?
\end{quotation}

\noindent Das System Pay2Mail kann als Antwort auf die Forschungsfrage verstanden werden. Die Triage wird durch die Verwendung von Gegenwerten grundlegend verändert und verringert den Aufwand des Empfängers. In der Nutzerevaluation gaben Empfänger an, dass sie die Triage zwar weiter selbst durchführen müssen, aber einen Anhaltspunkt für wichtige E-Mails haben und sich auf die Vorsortierung verlassen können. Zu beachten ist allerdings, dass sich der Aufwand für Empfänger und Absender erst verringert, wenn sie mit der Nutzung des Systems vertraut sind. Das gilt besonders für Absender, die Pay2Mail nur dann nutzen, wenn sie eine E-Mail priorisieren wollen. Ist dies nur selten der Fall, wird die Priorisierung stets mit einem größeren Aufwand verbunden sein. Da insbesondere der Aufwand des Empfängers verringert werden sollte, ist jener Aspekt zu vernachlässigen. Absender gaben in der Evaluation an einen Einfluss auf die Bearbeitung ihrer E-Mails nehmen zu können und ein Verständnis für Empfänger zu entwickeln, da sie das Aufkommen einsehen können. Auch dies bestätigt, dass Pay2Mail als Lösung des Problems der Triage und absendergesteuerten Priorisierung verstanden werden kann.

Abschließend ist zu erwähnen, dass das System allein nicht das übergeordnete Problem des E-Mail Overload lösen kann. In der Nutzerevaluation gaben Empfänger an, dass sie mit der automatischen Triage zwar eine größere Anzahl an E-Mails bearbeiten können, jedoch nimmt die Einsicht in das Aufkommen keinen direkten Einfluss auf die Anzahl der eingehenden E-Mails. Auch wenn Absender ein Verständnis für Bearbeitungszeiten entwickeln, ist nicht davon auszugehen, dass sie auch den Versand von E-Mails reduzieren. Zur Lösung des E-Mail Overload ist nicht nur eine Unterstützung der Empfänger, sondern auch eine nachhaltige Veränderung des Verhaltens von Absendern im Allgemeinen notwendig. 


\section{Ausblick}
\label{Ausblick}
Auch wenn Pay2Mail bereits eine gewisse Stabilität aufweist, sind mehrere Schritte und Weiterentwicklungen nötig, um die Anwendung zur Marktreife zu bringen. Die höchste Priorität hat die Anbindung realer Zahlungsdienstleister wie beispielsweise PayPal, Visa, Mastercard, Klarna oder das SEPA-Lastschriftverfahren. Zusätzlich zur Anbindung muss geprüft werden, inwieweit Rechnungen versendet werden müssen und ob Absenderdaten zur Zahlung gespeichert werden müssen. Eine Accountverwaltung für Absender sollte hier auch in Betracht gezogen werden, damit Absender, die das System häufiger nutzen, ihre Daten hinterlegen können.

Nach der Vervollständigung des Zahlungsmechanismus muss eine unmittelbare Integration in gängige E-Mail Clients durchgeführt werden. In der bisherigen Version des Systems müssen Absender Pay2Mail im Browser öffnen, eine E-Mail priorisieren und den Gegenwert-Header selbst in ihrem E-Mail Client setzen. Dieses Verfahren ist insbesondere auf mobilen Geräten aufwendig. Stattdessen müssen Plugins entwickelt werden, die die meistgenutzten E-Mail Clients erweitern. Ziel wäre es das UI von Pay2Mail nativ im Client darzustellen oder ein äquivalentes UI zu entwickeln, das die Features des Absenders abdeckt. Bei Letzterem müsste das Backend so erweitert werden, dass es als unabhängige REST API fungiert und universell für E-Mail Clients nutzbar ist. 

Um die User Experience zu verbessern und damit die potenzielle Zahl der Nutzer zu erhöhen, sollten die in der Nutzerevaluation vorgeschlagenen Features aus Kapitel \ref{Erkenntnisse_für_Pay2Mail} umgesetzt werden. Am wichtigsten ist die Verbesserung des Systems nach den Grundsätzen der Dialoggestaltung aus ISO 9241-110. Zusätzlich sollte der Zahlungsprozess so erweitert werden, dass die Zahlungen als Gebote erkennbar sind, inklusive der Benachrichtigung bei neuen Geboten und Erhöhung des eigenen Gebotes. Denkbar wären Deadlines für Gebote, sodass eine gewisse Sicherheit beim Absender entsteht, dass er nicht jederzeit überboten werden kann. Diese Deadlines könnten mit dem Wunsch nach einer Triage in Intervallen verbunden werden. So könnten beispielsweise Gebote jeweils bis 10:00 Uhr oder 14:00 Uhr abgegeben werden, daraufhin wird die Triage des Empfängers angepasst. Das hätte zum Einen den Vorteil, dass Absender sich einen festen Platz in der Triage sichern können, als auch dass Empfänger wissen, wann sich ihre Triage verändern kann. Ob dieses Verfahren jedoch verständlich ist und von Nutzern akzeptiert werden würde, müsste zunächst evaluiert werden.

Abschließend muss das System um professionelle Infrastruktur erweitert werden, um im Produktivbetrieb wartbar zu sein. Das beinhaltet ein strukturiertes Logging, Load Balancing von Datenbank und Server, sowie eine Erhöhung der Hardwarekapazitäten. Werden diese Schritte umgesetzt, kann Pay2Mail im Produktivbetrieb zur transparenten Darstellung und absendergesteuerten Priorisierung von E-Mail Warteschlangen verwendet werden.
