%!TEX root = ../main.tex
% Chapter 13

\chapter{Nutzerevaluation des Systems}
\label{Nutzerevaluation_des_Systems}

Im folgenden Kapitel wird Pay2Mail durch mehrere Nutzer evaluiert. Anders als in der Befragung zum Bezahl- und Tokensystem in Kapitel \ref{Nutzerbefragung_zum Bezahl-_und_Tokensystem} steht hier die Anwendung selbst im Fokus. Zudem ist die Anzahl der Teilnehmer deutlich geringer, dafür wird die Evaluation detaillierter durchgeführt. Die vollständigen Notizen und Antworten der Nutzerevaluation sind in Anhang \ref{Notizen_und_Antworten_der_Nutzerevaluation} zu finden.

\section{Planung der Evaluation}
\label{Planung_der_Evaluation}

Bei der Evaluation soll die Eignung als Lösung des Nutzungsproblems geprüft werden. Darüber hinaus sollen qualitative Faktoren, wie die intuitive Nutzung und die Verständlichkeit für Nutzer, erhoben werden. Die Eignung als Lösung kann durch die Befragung von Nutzern ermittelt werden, nachdem diese das System getestet haben. Qualitative Faktoren lassen sich hingegen nicht direkt erfragen. Daher wird die Evaluation in zwei Teilen durchgeführt. Vor der Evaluation wird den Teilnehmern das Problemfeld der Triage aus Kapitel \ref{Problemfeld_und_Folgen} verkürzt erklärt.

Im ersten Teil wird eine \textit{in vivo}-Evaluation durchgeführt. Das bedeutet, dass die Teilnehmer in einem möglichst realitätsnahen Szenario gebeten werden das System zu nutzen. Dazu werden ihnen, je nach Rolle, Aufgaben gestellt, die sie bewältigen müssen. Die Aufgaben entsprechen den Use Cases aus Kapitel \ref{Use_Cases}. Dabei sollen sie ihre Handlungsschritte, Gedänkengänge und Anmerkungen stets erläutern. Diese Erläuterungen werden festgehalten, jedoch nicht kommentiert oder bewertet. Sie dienen als Grundlage zur Erhebung der qualitativen Faktoren.

Nach Erledigung der Aufgaben werden den Teilnehmern Fragen gestellt, die die Eignung des Systems ermitteln sollen. Darüber hinaus können die Teilnehmer abschließend selbst Verbesserungsvorschläge für das System äußern. Die folgenden Fragen werden gestellt:

\begin{itemize}
\item \textit{An Absender gestellt:} Haben Sie das Gefühl, dass Sie E-Mails effektiver nutzen können? Warum (nicht)?
\item \textit{An Absender gestellt:} Bietet Ihnen die Einsicht in das E-Mail Aufkommen einen Vorteil im Vergleich zu \textit{normalen} Posteingängen? Warum (nicht)?
\item \textit{An Empfänger gestellt:} Unterstützt Sie die automatische Triage bei der Bearbeitung Ihrer E-Mails? Warum (nicht)?
\item \textit{An Empfänger gestellt:} Unterstützt Sie die Definition von Regeln bei der Nutzung von Pay2Mail? Warum (nicht)?
\item Löst Pay2Mail für Sie das Problem des E-Mail Overload? Warum (nicht)?
\item Was würden Sie an Pay2Mail verbessern? 
\end{itemize}

\noindent Die Fragen sind so formuliert, dass sie zunächst die generelle Meinung der Befragten (mit Ja/Nein) und daraufhin entsprechende Kausalitäten erheben. Das hat den Vorteil, dass die Fragen nicht suggestiv erscheinen und das Ergebnis nicht verfälschen.

Aufgrund des Umfangs der Evaluation werden zwei Empfänger und zwei Absender ausgewählt. Diese Größe ist nicht repräsentativ, bietet jedoch die Möglichkeit je zwei Sichtweisen auf das System zu erhalten. Bei den Teilnehmern handelt es sich um Personen, die E-Mails mindestens mehrfach wöchentlich nutzen. Nach Möglichkeit sollen die Teilnehmer E-Mail Overload bereits selbst erfahren haben oder zumindest mit dem Konzept vertraut sein. Es wird je eine Person mit höherem und eine mit niedrigerem Einkommen ausgewählt, um die verschiedenen Ansichten zum Ansatz zu Pay to Win, wie in Kapitel \ref{Ergebnisse_der_Befragung} ermittelt, zu berücksichtigen. Zur Wahrung der Anonymität werden die Absender mit \textit{A1} und \textit{A2} und die Empfänger mit \textit{E1} und \textit{E2} gekennzeichnet. Zusätzlich werden die Stammdaten \textit{Geschlecht}, \textit{Alter}, \textit{Jahresnettoeinkommen} und \textit{Beruf} aufgenommen.

\section{Ergebnisse der Evaluation}
\label{Ergebnisse der Evaluation}
Die Absender sollen zunächst das Aufkommen eines Empfängers einsehen und daraufhin eine E-Mail so priorisieren, dass sie innerhalb von drei Tagen bearbeitet wird. Dazu gaben beide Teilnehmer die Adresse des Empfängers ein und sahen sich zunächst das Aufkommen an. Hier merkte A2 an, dass der Unterschied zwischen Priorisierung und Aufkommen zu Beginn nicht verständlich ist. Außerdem fehlte A1 bei der Einsicht in das Aufkommen die Bestätigung, dass die E-Mail Adresse richtig eingetragen wurde, da sie unter \texttt{/capacities} nicht mehr zu sehen ist. Nach der Einsicht in das Aufkommen entschieden sich beide Teilnehmer die E-Mail zu priorisieren. A1 fehlte eine Anleitung zur Zahlung, die Minimal- und Maximalwerte beschreibt. Ebenfalls wurde kritisiert, dass man bei Klick auf \textit{E-Mail priorisieren} erst auf die Tabelle mit den bisherigen Zahlungen und nicht auf die eigentliche Priorisierung weitergeleitet wird. Auf \texttt{/priority/new} wird die E-Mail Adresse nicht aus den vorherigen Schritten übernommen, merkte A1 an. A2 fand sich in der Priorisierung auf Anhieb zurecht und hatte keine Anmerkungen im Prozess. Auch das Prinzip des Gegenwert-Headers wurde verstanden, allerdings wurde sich eine Anleitung zum Versenden einer E-Mail gewünscht. A1 verstand das Prinzip des Gegenwert-Headers ebenfalls und war in der Lage eine entsprechende E-Mail zu versenden. Nach Erhalt des Headers wurde jedoch angemerkt, dass dem System eine permanente Navigation fehlt, sodass Nutzer nicht genau wissen wo im Prozess sie sich befinden.

Die Empfänger erhielten Daten eines Testnutzers und sollten sich im System registrieren und daraufhin eine E-Mail im Postfach mit dem Betreff \textit{Bitte um Rückruf} finden. Abschließend sollte diese E-Mail erledigt werden. Die Ausnahmeregeln wurden den Teilnehmer daraufhin vorgestellt, jedoch nicht in vivo evaluiert, da sie keine Kernfunktionalität des Systems abdecken. Die Registrierung wurde von beiden Teilnehmer auf Anhieb gefunden und ohne Komplikationen durchgeführt. Hierbei ist jedoch zu erwähnen, dass nicht jeder Nutzer seine IMAP-Anmeldedaten kennt oder weiß, wie er sie erhält. Daher kann die Registrierung je nach Kenntnisstand der Nutzer und E-Mail Provider zu Problemen führen. Da die Nutzer nach der Registrierung direkt angemeldet sind, entfällt der Login. Da dieser dem Formular zur Registrierung ähnelt, kann er vernachlässigt werden. Die Teilnehmer beschrieben den Posteingang als leicht verständlich und intuitiv, was sich auf Gemeinsamkeiten mit ihren E-Mail Clients bezieht. E2 merkte an, dass die Anzeige des gezahlten Betrags für den Empfänger keine Relevanz besitzt, da der Posteingang bereits sortiert ist. Außerdem könne dies Druck beim Empfänger auslösen mehr E-Mails zu bearbeiten, da sie zwar noch nicht fällig sind, aber mit einem hohen Gegenwert versehen worden sind. Somit würde die automatische Triage nicht vollständig nutzbar sein, da die subjektive Wahrnehmung des Empfängers dazu führt, dass E-Mails unter Umständen vorgezogen werden. Laut E2 löst sich dieses Problem, wenn der gezahlte Betrag aus dem Posteingang entfernt wird. E1 hinterfragt den sozialen Aspekt von Pay2Mail, da Personen mit geringerem Einkommen keine Möglichkeit haben mit finanziell stärkeren Absendern zu konkurrieren. Diese Argumentation deckt sich mit den Vermutungen aus Kapitel \ref{Auswahl_eines_Bezahl-_und_Tokensystems} und wird in Kapitel \ref{Reflektion_und_Bewertung} diskutiert. E1 merkte jedoch an, dass Ausnahmeregeln die Möglichkeit bieten, diese soziale Ungerechtigkeit zu lösen. Vorausgesetzt ist, dass Empfänger bereits zu Beginn der Nutzung von Pay2Mail wissen, welche Absender diese Unterstützung benötigen. Als sehr relevant stuft auch E2 die Ausnahmeregeln ein. So können E-Mails, die wichtig sind, aber nicht priorisiert wurden, vorgezogen werden. Die automatische Triage wird mit Ausnahmeregeln so angereichert, dass sie den realen Prioritäten entspricht. Jedoch merkt auch E2 an, dass dieser anfängliche Schritt zeitaufwändig ist und sich nicht lohnt, wenn die Anwendung nicht mittel- bis langfristig intensiv genutzt wird.  

Abschließend beantworteten alle Teilnehmer die in \ref{Planung_der_Evaluation} formulierten Fragen. Die Absender sehen Pay2Mail als Möglichkeit an die Nutzung von E-Mails effektiver zu gestalten. Allerdings wird die Effizienz in Frage gestellt, da der Prozess zeitaufwändiger ist als der herkömmliche Versand. Mit einer gewissen Nutzungsdauer und Routine wird jedoch angenommen, dass der Prozess effizienter wird. Die Einsicht in das Aufkommen wird ebenfalls als optionaler und hilfreicher Schritt angesehen. Man kann sich als Absender entscheiden, ob man am Aufkommen des Empfängers interessiert ist oder ob man lediglich eine E-Mail versenden will, ohne den Bearbeitungszeitpunkt zu kennen. Ersteres hilft laut A2 dabei ein \emph{\glqq Gefühl für den Absender\grqq{}} zu entwickeln bezogen auf die generelle Antwortzeit und Auslastung. Die Empfänger bewerten die automatische Triage als Unterstützung der Bearbeitung ihrer E-Mails. Allerdings sehen sie sie nicht als alleiniges Mittel zur Priorisierung von E-Mails, sondern nur als Orientierung an. Stattdessen erachten sie die Ausnahmeregeln als ein äußerst wichtiges Instrument zur Individualisierung der Triage an. Jedoch ist zu beachten, dass diese Individualisierung zu Beginn zeitaufwändig ist, ferner werden sich weitere Felder zur Definition von Regeln gewünscht. 

Pay2Mail wird sowohl von Absendern, als auch Empfängern als Unterstützung gegen das Problem des E-Mail Overloads, nicht aber als alleinige Lösung, angesehen. Laut Absendern wird der generelle Prozess verbessert, wobei es für reichere Absender möglich ist weniger wichtige E-Mails als sehr relevant darzustellen, indem sie sehr hohe Gegenwerte setzen. Die Empfänger sind der Meinung, dass die transparente Einsicht in das Aufkommen dafür sorgt, dass Absender ein Verständnis für längere Antwortzeiten aufbauen. Da E-Mail Overload sich nicht nur auf die Anzahl der offenen E-Mails, sondern auch auf deren Dringlichkeit bezieht, kann ein solches Verständnis den Overload bereits entsprechend verringern. Hinzu kommt, dass die Empfänger die Verantwortlichkeit der Triage zumindest teilweise den Absendern überlassen und somit nur bedingt Rechenschaft schuldig sind, falls eine Bearbeitung längere Zeit in Anspruch nimmt. Die Verbesserungsvorschläge unterscheiden sich zwischen den Teilnehmergruppen. Die Absender fordern eine bessere Verständlichkeit und Steuerbarkeit des Systems durch Navigation, Erklärungstexte und den Hinweis, dass es sich bei Pay2Mail um ein Bietersystem handelt, sodass die eigene E-Mail überboten und somit ent-priorisiert werden kann. Es soll Absendern außerdem möglich sein ihr Gebot, also ihren Gegenwert, zu erhöhen, sobald sie überboten wurden. Dazu müssen sie informiert werden, sobald sich der Bearbeitungszeitpunkt ihrer E-Mail verändert. Die Empfängergruppe hingegen fordert Anpassungen der Triage. Es soll möglich sein finanziell schwächere Absender so zu stärken, dass sie mit anderen Absender konkurrieren können. Außerdem soll die Flexibilität der Triage eingeschränkt werden. Das bedeutet, dass sich die Reihenfolge der zu bearbeitenden E-Mails nur zu bestimmten Zeitpunkten ändern kann. Das hat den Vorteil, dass Empfänger keine E-Mails bearbeiten, die durch eine Anpassung der Triage bereits wieder weniger Priorität haben. Ob diese Zeitpunkte nur den Empfängern oder auch Absendern mitgeteilt werden sollen, wurde von den Teilnehmern der Evaluation offen gelassen.

\section{Erkenntnisse für Pay2Mail}
\label{Erkenntnisse_für_Pay2Mail}

Die zuvor erhobenen Antworten und Aussagen zeigen auf, dass das Konzept von Pay2Mail ein Potenzial aufweist sinnvoll gegen E-Mail Overload verwendet zu werden. Das Prinzip von Priorisierung durch Gegenwerte, also Pay to Win, wurde angenommen, wenn auch mit Einschränkungen hinsichtlich der sozialen Gerechtigkeit. Außerdem wurde das System als weitestgehend intuitiv und verständlich wahrgenommen. Der Wunsch nach einer besseren Steuerbarkeit und Selbstbeschreibungsfähigkeit nach \cite{ISO9241-110} zeigt jedoch auf, dass weitere Verbesserungen am System nötig sind. Darüber hinaus muss das Zahlungssystem so erweitert werden, dass eine Priorisierung als Gebot gekennzeichnet wird. Dies beinhaltet eine Benachrichtigung bei neuen und größeren Geboten, sowie die Möglichkeit sein eigenes Gebot zu erhöhen. Beispiel für eine solche Mechanik ist das Verkaufsportal \textit{eBay}. Zusätzlich zu den Geboten soll sich die Triage nur zu bestimmten Zeitpunkten verändern können, um Empfängern eine gewisse Stabilität in der Bearbeitung ihrer E-Mails zu geben. Inwiefern sich Gebote und die Zeitpunkte der \textit{Triageaktualisierungen} kombinieren lassen, müssten folgende Evaluationen und Befragungen erheben. Funktionalitäten und Anforderungen zur Weiterentwicklung von Pay2Mail hin zu einem marktreifen Produkt werden im Ausblick in Kapitel \ref{Ausblick} thematisiert.