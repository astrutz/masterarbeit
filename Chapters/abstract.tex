%!TEX root = ../main.tex

\chapter*{Abstract}
\addchaptertocentry{Abstract}
E-Mails haben sich in den letzten drei Jahrzehnten zu einem der wichtigsten Kommunikationsmittel etabliert. So stieg die Anzahl der E-Mails in Deutschland innerhalb von 18 Jahren um das 26-fache, Spam-Mails exklusive. Durch das erhöhte E-Mail Aufkommen sind Empfänger gezwungen eine Sortierung und Priorisierung ihrer Nachrichten vorzunehmen, auch Triage genannt. Trotz der Triage sind Empfänger überfordert, wodurch Stress und eine psychische Belastung entstehen, bekannt als E-Mail Overload. Um dieses Problem zu lösen wurde als Ziel dieser Arbeit die Implementation eines Systems definiert, welches den Aufwand der Triage für Empfänger verringert. Nach einer Forschungs- und Marktrecherche wurde sich für die Erweiterung von SMTP um ein E-Mail Header-Feld entschieden. Innerhalb dieses Feldes können Absender ihre E-Mails mit einem finanziellen Gegenwert versehen und sie so priorisieren, angelehnt an \cite{Turner2003}. In einer nicht-repräsentativen Befragung gaben 61,5\% der Teilnehmer an, dass sie ein solches Zahlungssystem mit Echtgeld verwenden würden. Die Anwendung \textit{Pay2Mail} (\url{https://pay2mail.org}) wurde entwickelt, die es ermöglicht sowohl E-Mails mit Gegenwerten zu priorisieren, als auch als Empfänger einen priorisierten Posteingang zu erhalten. Eine anschließende Nutzerevaluation ergab, dass Pay2Mail zwar das Problem des E-Mail Overload nicht löst, aber Empfänger dabei unterstützt E-Mails effizienter bearbeiten zu können. Kritisiert wurde die fehlende soziale Gerechtigkeit des Systems.
\\ \\
Over the last three decades, e-mails have become one of the most important means of communication. For example, the number of e-mails in Germany increased 26-fold within 18 years, excluding spam mails. Due to the increased volume of e-mails, recipients are forced to sort and prioritize their messages, also called Triage. Despite the triage, recipients are overwhelmed, causing stress and a psychological burden known as email overload. To solve this problem, the goal of this thesis was defined as the implementation of a system that reduces the triage effort for recipients. After research and market investigation, it was decided to add an email header field to SMTP. Within this field, senders can attach a monetary equivalent to their e-mails and thus prioritize them, adapted from \cite{Turner2003}. In a non-representative survey, 61.5\% of participants indicated that they would use such a payment system with real money. The \textit{Pay2Mail} application (\url{https://pay2mail.org}) was developed to both prioritize emails with counter values and to receive a prioritized inbox as a recipient. A subsequent user evaluation showed that Pay2Mail does not solve the problem of e-mail overload, but it does help recipients to process e-mails more efficiently. The lack of social justice in the system was criticized.