\newglossaryentry{technischeemail}{name={Technische E-Mail}, description={E-Mail, welche von Systemen automatisch versendet wird}}
\newglossaryentry{triage}{name={Triage}, description={\emph{\glqq[...] Prozess, bei dem unbearbeitete E-Mails durchgesehen und entschieden wird, was damit zu tun ist.\grqq{}} \citep[S. 325]{Sarrafzadeh2019}}}
\newglossaryentry{emailoverload}{name={E-Mail Overload}, description={Assoziation von Stress mit einem erhöhten Aufkommen von E-Mails \citep[S. 253]{Thomas2006}}}
\newglossaryentry{_hrv}{name={Heart Rate Variability}, description={Variation des Intervalls zwischen Herzschlägen, kardiologisches Anzeichen für Stress und Depressionen \citep[S. 881]{Vrijkotte2000}}}
\newglossaryentry{_smtp}{name={Simple Mail Transport Protocol}, description={Protokoll, welches den Versand von E-Mails zwischen MUA und MTA, sowie MTA und MDA ermöglicht \citep[S. 1]{RFC821}}}
\newglossaryentry{_pop3}{name={Post Office Protocol 3}, description={Protokoll zum Abrufen von E-Mails vom MUA \citep[S. 2]{RFC1939}}}
\newglossaryentry{_imap}{name={Internet Message Access Protocol}, description={Protokoll zum Abrufen von E-Mails vom MUA mit erweiterten Funktionen gegenüber POP3 \citep[S. 6 f.]{RFC3501}}}
\newglossaryentry{_ctss}{name={Compatible Time-Sharing System}, description={Betriebssystem, welches es ermöglicht einen Rechner auf verschiedene Benutzer aufzuteilen zur gleichzeitigen Nutzung \citep[S. 2]{Corbato1963}}}
\newglossaryentry{_arpanet}{name={Advanced Research Projects Agency Network}, description={Dezentrales Computernetzwerk, mit welchem allgemeine Kommunikation ermöglicht wurde. Gilt als Vorläufer des Internets \citep[S. 426 ff.]{Marill1966}}}
\newglossaryentry{_mua}{name={Mail User Agent}, description={Client der E-Mail Architektur, ermöglicht das Verfassen, Versenden, Empfangen und Verwalten \citep[S. 142 f.]{Kurose2014}}}
\newglossaryentry{_mta}{name={Mail Transfer Agent}, description={Mailserver, kann via SMTP kommunizieren \citep[S. 142 f.]{Kurose2014}}}
\newglossaryentry{_mime}{name={Multipurpose Internet Mail Extensions}, description={Erweiterungen des Datenformats für E-Mails um weitere Zeichensätze und Anhänge \citep[S. 3 f.]{RFC2045}}}
\newglossaryentry{_dsn}{name={Delivery Status Notification}, description={Zustellbestätigungen für Mails, ausgestellt durch MDA \citep[S. 7 f.]{RFC3463}}}
\newglossaryentry{_mdn}{name={Message Disposition Notification}, description={Empfangsbestätigungen für Mails, ausgestellt durch MUA \citep[S. 3 f.]{RFC3503}}}
\newglossaryentry{_tls}{name={Transport Layer Security}, description={Mechanismus zur Verschlüsselung von Daten im Internet durch den Diffie-Hellman-Schlüsselaustausch \citep[S. 95 f.]{RFC8446}}}
\newglossaryentry{_smime}{name={Secure/Multipurpose Internet Mail Extensions}, description={Erweiterung von MIME zur Verschlüsselung von E-Mails \citep[S. 5 ff.]{RFC8551}}}
\newglossaryentry{pay2win}{name={Pay to Win}, description={Monetarisierung von Videospielen, die kostenlos oder kostenpflichtig sind durch Mikrotransaktionen \citep[S. 18]{Tomic2019}}}
\newglossaryentry{lootbox}{name={Lootbox}, description={Kaufbare Pakete in Videospielen, welche nicht vorhersagbaren Inhalt enthalten \citep[S. 20]{Tomic2019}}}
\newglossaryentry{_lcp}{name={Lightweight Currency Protocol}, description={Vorgeschlagenes Protokoll zur Abwicklung von Zahlungen im Internet über organisationsspezifische leichtgewichtige Währungen \citep[S. 3]{Turner2003}}}
\newglossaryentry{pay2mail}{name={Pay2Mail}, description={System zur transparenten Darstellung und absendergesteuerten Priorisierung von E-Mail Warteschlangen, welches im Rahmen dieser Arbeit implementiert wird}}
\newglossaryentry{bearbeitungsleistung}{name={Bearbeitungsleistung}, description={Anzahl der E-Mails, die ein Empfänger in Pay2Mail täglich bearbeiten kann, verbindlich für die Priorisierung mit Gegenwert}}
\newglossaryentry{_uuid}{name={Universally Unique Identifier}, description={Generierte 128 Bit-Nummer, die zur Identifikation von Objekten genutzt wird und äußerst selten zu Duplikaten führt \citep[S. 2 f.]{RFC4122}}}
\newglossaryentry{_dml}{name={Data Manipulation Language}, description={Übergeordnete Gruppe der Befehle zur Bearbeitung von Daten in einer Datenbank \citep[S. 9]{Chatham2012}}}
\newglossaryentry{node.js}{name={Node.js}, description={Laufzeitumgebung zur serverseitigen eventbasierten Ausführung von JavaScript \citep{OpenJSFoundation2022}}}
\newglossaryentry{laravel}{name={Laravel}, description={Auf PHP aufbauendes Framework nach dem MVC-Pattern \citep{Otwell2022}}}
\newglossaryentry{rubyonrails}{name={Ruby on Rails}, description={Auf Ruby aufbauendes Framework nach dem MVC-Pattern \citep{Hansson2022a}}}
\newglossaryentry{_dry}{name={Don't Repeat Yourself}, description={Strikte Vermeidung von dupliziertem Code \citep{Hansson2022a}}}
\newglossaryentry{codeoverconvention}{name={Code Over Convention}, description={Verzicht auf zwingende Konfigurationen durch hilfreiche Standardwerte, auch \textit{reasonable defaults} genannt \citep{Hansson2022a}}}
\newglossaryentry{embeddedruby}{name={Embedded Ruby}, description={Templatesprache, mit der in Ruby on Rails HTML mit Anwendungslogik verbunden werden kann \citep{Hansson2022a}}}
\newglossaryentry{gem}{name={Gem}, description={Externe Module für Ruby on Rails, die Standardfunktionalitäten erweitern \citep{Hansson2022a}}}
\newglossaryentry{githubactions}{name={GitHub Actions}, description={CI/CD-Umgebung, die direkt in GitHub integriert ist \citep{GitHub2022}}}
\newglossaryentry{bootstrap}{name={Bootstrap}, description={Frontend-Library, die es Entwicklern ermöglicht vorgefertigte Komponenten zu verwenden, die nach einem Styleguide entwickelt wurden \citep{Twitter2022}}}
