%\newglossaryentry{_2fa}{name={Zwei-Faktor-Authentifizierung}, description={tbd}}
\newglossaryentry{technischeemail}{name={Technische E-Mail}, description={E-Mail, welche von Systemen automatisch versendet wird}}
\newglossaryentry{triage}{name={Triage}, description={\glqq[...] Prozess, bei dem unbearbeitete E-Mails durchgesehen und entschieden wird, was damit zu tun ist.\grqq{} \citep[S. 325]{Sarrafzadeh2019}}}
\newglossaryentry{emailoverload}{name={E-Mail Overload}, description={Assoziation von Stress mit einem erhöhten Aufkommen von E-Mails \citep[S. 253]{Thomas2006}}}
\newglossaryentry{_hrv}{name={Heart Rate Variability}, description={Variation des Intervalls zwischen Herzschlägen, kardiologisches Anzeichen für Stress und Depressionen \citep[S. 881]{Vrijkotte2000}}}
%\newglossaryentry{_ascii}{name={American Standard Code for Information Interchange}, description={tbd \citep[S. 325]{Sarrafzadeh2019}}}
%\newglossaryentry{_osi}{name={Open Systems Interconnection}, description={tbd \citep[S. 325]{Sarrafzadeh2019}}}
\newglossaryentry{_smtp}{name={Simple Mail Transport Protocol}, description={Protokoll, welches den Versand von E-Mails zwischen MUA und MTA, sowie MTA und MDA ermöglicht \citep[S. 1]{RFC821}}}
\newglossaryentry{_pop3}{name={Post Office Protocol 3}, description={tbd \citep[S. 325]{Sarrafzadeh2019}}}
\newglossaryentry{_imap}{name={Internet Message Access Protocol}, description={tbd \citep[S. 325]{Sarrafzadeh2019}}}
%\newglossaryentry{_tcp}{name={Transmission Control Protocol}, description={tbd \citep[S. 325]{Sarrafzadeh2019}}}
%\newglossaryentry{_ip}{name={Internet Protocol}, description={tbd \citep[S. 325]{Sarrafzadeh2019}}}
\newglossaryentry{_ctss}{name={Compatible Time-Sharing System}, description={Betriebssystem, welches es ermöglicht einen Rechner auf verschiedene Benutzer aufzuteilen zur gleichzeitigen Nutzung \citep[S. 2]{Corbato1963}}}
\newglossaryentry{_arpanet}{name={Advanced Research Projects Agency Network}, description={Dezentrales Computernetzwerk, mit welchem allgemeine Kommunikation ermöglicht wurde. Gilt als Vorläufer des Internets \citep[S. 426 ff.]{Marill1966}}}
%\newglossaryentry{_ftp}{name={File Transfer Protocol}, description={tbd \citep[S. 325]{Sarrafzadeh2019}}}
%\newglossaryentry{tld}{name={Top Level Domain}, description={tbd \citep[S. 325]{Sarrafzadeh2019}}}
%\newglossaryentry{ipadresse}{name={IP-Adresse}, description={tbd \citep[S. 325]{Sarrafzadeh2019}}}
\newglossaryentry{_mua}{name={Mail User Agent}, description={tbd \citep[S. 325]{Sarrafzadeh2019}}}
\newglossaryentry{_mta}{name={Mail Transfer Agent}, description={tbd \citep[S. 325]{Sarrafzadeh2019}}}
\newglossaryentry{_mime}{name={Multipurpose Internet Mail Extensions}, description={tbd \citep[S. 325]{Sarrafzadeh2019}}}
%\newglossaryentry{_rfc}{name={Request for Comments}, description={tbd \citep[S. 325]{Sarrafzadeh2019}}}
\newglossaryentry{_dsn}{name={Delivery Status Notification}, description={tbd \citep[S. 325]{Sarrafzadeh2019}}}
\newglossaryentry{_mdn}{name={Message Disposition Notification}, description={tbd \citep[S. 325]{Sarrafzadeh2019}}}
\newglossaryentry{_tls}{name={Transport Layer Security}, description={tbd \citep[S. 325]{Sarrafzadeh2019}}}
\newglossaryentry{_smime}{name={Secure/Multipurpose Internet Mail Extensions}, description={tbd \citep[S. 325]{Sarrafzadeh2019}}}
\newglossaryentry{pay2win}{name={Pay to Win}, description={Monetarisierung von Videospielen, die kostenlos oder kostenpflichtig sind durch Mikrotransaktionen \citep[S. 18]{Tomic2019}}}
\newglossaryentry{lootbox}{name={Lootbox}, description={Kaufbare Pakete in Videospielen, welche nicht vorhersagbaren Inhalt enthalten \citep[S. 20]{Tomic2019}}}
\newglossaryentry{_lcp}{name={Lightweight Currency Protocol}, description={Vorgeschlagenes Protokoll zur Abwicklung von Zahlungen im Internet über organisationsspezifische leichtgewichtige Währungen \citep[S. 3]{Turner2003}}}
\newglossaryentry{pay2mail}{name={Pay2Mail}, description={System zur transparenten Darstellung und absendergesteuerten Priorisierung von E-Mail Warteschlangen, welches im Rahmen dieser Arbeit implementiert wird}}
\newglossaryentry{bearbeitungsleistung}{name={Bearbeitungsleistung}, description={Anzahl der E-Mails, die ein Empfänger in Pay2Mail täglich bearbeiten kann, verbindlich für die Priorisierung mit Gegenwert.}}